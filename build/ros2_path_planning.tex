
\documentclass[journal]{IEEEtran}

\usepackage{biblatex}

\usepackage[fleqn]{amsmath}
\usepackage{amssymb}
\usepackage{graphicx}
\usepackage{cancel}
\usepackage{tabularx}

\usepackage{caption}
% \usepackage{subcaption}

\usepackage{subfig}
\addbibresource{./citations.bib}

\graphicspath{ {../images}}

\usepackage{hyperref}
\hypersetup{
colorlinks=false,
linkcolor=blue,
filecolor=magenta,
    urlcolor=blue,
}
\urlstyle{same}



\begin{document}

\title{Path Planning For Autonomous Vehicles \\ Dr. Assadian - Future Mobility Lab}


\author{Jonathan~Dorsey \IEEEmembership{Member: No Sleep Club: est. 2017} \\  \url{https://github.com/JonnyD1117/ros2-py-path-planning-server}}


% The paper headers
\markboth{Journal of Graduate School Assignments, March~2023}%

\maketitle



% As a general rule, do not put math, special symbols or citations
% in the abstract or keywords.
\begin{abstract}
  The objective of this project is.....???
\end{abstract}

% Note that keywords are not normally used for peerreview papers.
\begin{IEEEkeywords}
	A*, path planning, mobile robots, autonomous vehicles
\end{IEEEkeywords}


\IEEEpeerreviewmaketitle

\section{Introduction}

\IEEEPARstart{T}{he} world of robotics is full of constraints, demands, and performance trade-offs that humans handle naturally on a daily basis.

\section{Path Planning for Mobile Robots}


\subsection{ Motivation}

\section{Path Planning Algorithms}

\subsection{Graph Search}

\subsubsection{A*}

\subsubsection{Dijstra's Shortest Path}

\subsection{Rapidly Exploring Random Trees}

\subsubsection{RRT}

\subsubsection{RRT*}

\subsection{Probabilistic Road Maps}


\section{Path Smoothing}

\section{Way Point Generation}

\section{Acknowledgment}

\begin{thebibliography}{00}
\bibitem{b2} J. Clerk Maxwell, A Treatise on Electricity and Magnetism, 3rd ed., vol. 2. Oxford: Clarendon, 1892, pp.68--73.
\end{thebibliography}
\vspace{12pt}

\end{document}