\section{The Path Planning Problem} 

As it applies to mobile robotics, the path planning problem can be stated as, the task of generating a feasibly traversable sequence of waypoints (aka a path) between a starting position and a target destination. The creation of this path is possibly subject to heuristics, a cost-function, or set of constraints and in its' most general form, allows for the inclusion of dynamics obstacles, in the environment. 

It should be immediately obvious that this definition describes a very broad class of physical environments, path planning algorithms, and path generation criteria. However, with the inclusion of a few notable exceptions, most path planners that are described by the previous definition will require a solution that plans in realtime, to handle the dynamic elements within an environment. This is not always desirable, reasonable, or even realizable depending on a robots intended function and available computational resources. Often, assumptions like these need to be relaxed and simplified as a compromise to reaching a feasible solution.

\subsection{Offline Planning for Static Environments}

In many cases, the assumption of a static environment is not only a sufficient model but also simplifies the problem statement significantly unlocking the ability to use more flexible and lightweight path planners. While applying this assumption certainly limits the scope of the path planners utility, the scenario of planning through a static environment is frequently a very useful benchmark as a means of comparing the relative merits of different algorithms and planning mechanisms since the environment is time-invariant. As the objective of this paper is to present and compare a handful of fundamental path planning algorithms, the rest of this paper makes the assumption that path planning is occurring in a static environment.