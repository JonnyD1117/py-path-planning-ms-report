\section{The Path Planning Problem} 

As it applies to mobile robotics, the path planning problem can be stated as, the task of generating a feasibly traversable sequence of waypoints (aka a path) between a starting position and a target destination. The creation of this path is possibly subject to finite compute resources, a cost-function, and/or a set of system constraints. Additionally, in its' most general form, path planning allows for the inclusion of dynamics obstacles, in the environment. 

It should be immediately obvious that this definition describes a very broad class of physical environments, possible algorithms, and path generation criteria. However, with the inclusion of a few notable exceptions, most path planners that are described by the previous definition will require a solution that plans in realtime, to handle the dynamic elements within an environment. This is not always desirable, reasonable, or even realizable. Depending on the intended function and computational resources available, these assumptions often need to be relaxed and simplified to reach a feasible solution

\subsection{Offline Planning for Static Environments}

One of the most common simplifying assumptions, within the domain of path planning, is to assume that the world the mobile robot is operating within is completely static. Frequently, this assumption is sufficient to model large complex environment. This not only permits a class of faster and less resource intensive planning algorithms, but also provides a simple and reliable foundational layer that obstacle detection and avoidance algorithms can build on to handle dynamic elements in a local (instead of a global) sense, allowing simple programs to build complex behaviors together in a modular fashion instead of building large monolithic solutions. 

Additionally, static planners are frequently utilized to quantify the statistical performance of an algorithm or to understand the limitations that could arise from the application of planning in real-world edge conditions. 

As the objective of this paper is to present and compare a handful of fundamental path planning algorithms, the rest of this paper makes the assumption that path planning is occurring in a static environment.