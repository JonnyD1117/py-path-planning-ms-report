\section{The Path Planning Problem} 
The general path planning problem focuses on the task of generating a feasible and traversable sequence of waypoints (e.g. a path) between a starting position and a target destination. The creation of this path is possibly subject to finite compute resources, a cost-function, and/or a set of system constraints. Additionally, in its' most general form, path planning allows for the inclusion of dynamics obstacles, in the environment \cite{Lav06}. 

It should be immediately obvious that this definition encompasses a broad class of physical environments, algorithms, and path generation criteria. While the gold standard for path planning is real-time path generation in unknown dynamic environments, achieving this goal is frequently not desirable or realizable for a system to reach given the objectives, constraints, and limitations it must operate within. Depending on the intended function and computational resources available, planning assumptions often need to be relaxed or simplified to obtain a viable solution.

\subsection{Path Planning in Mobile Robotics}
It should be noted that path planning possesses many applications beyond those of autonomous vehicles and mobile robotics, such as in video games, and end effector manipulation tasks. While the general premise of generating a path between two configurations is the same, these applications frequently have extremely different limitations, constraints, and performance requirements even though the underlying theory transcends the specifics of the application.

As the intended application of this paper is to apply path planning techniques to real-world autonomous vehicles, it is important to understand the limitations, criteria, and specifications that this application usually demands. The following is a list of a few of the most common constraints.


\begin{enumerate}
    \item Real-time performance 
    \item High bandwidth sensor inputs (e.g. Lidars \& Cameras)
    \item Obstacle avoidance 
    \item Limited onboard computation
    \item Dynamic/Uncertain environments
    \item Needs to both localize within and map its environment.
\end{enumerate}


\subsection{Offline Planning for Static Environments}

By far, the most common simplification applied to path planning is to assume that the world the mobile robot is operating within is completely static. This implies that so long as the map the vehicle is using to plan is correct and a feasible path exists, planning can be done offline since the environment is assumed to be time invariant. This assumption possesses several attractive properties concerning both planning formulation as well as the planner implementation and performance.

From the perspective of formulating the planning problem, this assumption significantly reduces scope and complexity by removing all mapping and environmental uncertainty at runtime. By eliminating the need to handle dynamic obstacles or mapping uncertainty, this assumption lets roboticists get to the heart of the planning and to explore the theory of planning a path from point "a" to point "b", without the messy engineering details that crop up in practice. As the objective of this paper is to present and compare a handful of fundamental path planning algorithms, the rest of this paper makes the assumption that path planning is occurring in a static environment.