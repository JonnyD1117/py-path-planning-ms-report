This work introduces the problem of path planning for mobile autonomous vehicles, in static environments. It focuses on the practical context surrounding path planning such as occupancy grid maps, simultaneous localization and mapping (SLAM), and graph generation. After providing a broad overview of the general problem context, this paper surveys several of the most common and widely used planning algorithms. These algorithms include A*, Dijkstra's shortest path, Rapidly Exploring Random Trees, and Probabilistic Road Maps. Each of these algorithms has previously been the stateoftheart, in path planning, for mobile robots over the years and each continues to provide the theoretical foundational that many modern path planners are often synthesized from. This paper concludes with a case study covering the realworld performance, capabilities, and limitations for each of the aforementioned algorithms under a scenario of path planning for autonomous vehicles in known environments.