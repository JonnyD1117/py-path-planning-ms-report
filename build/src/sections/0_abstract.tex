This work introduces the problem of path planning for mobile autonomous vehicles, in static environments. It focuses on the practical context surrounding path planning such as occupancy grid maps, simultaneous localization and mapping (SLAM), and graph generation culminating in a survey of several fundamental planning algorithms, their pros and conns concerning both performance and computational efficiency. These algorithms include A*, Dijkstra's shortest path, Rapidly Exploring Random Trees, and Probabilistic Road Maps. Each of these algorithms has previously been the stateoftheart in path planning for mobile robots, and provides the theoretical foundational upon which modern path planners are often synthesized. This paper concludes by identifying the strengths and weaknesses of each planning algorithm, with suggestions concerning their most appropriate applications, and finally concludes by a discussion of possible extensions of path planners into dynamic environments by the inclusion of a local planner.  
